% Options for packages loaded elsewhere
\PassOptionsToPackage{unicode}{hyperref}
\PassOptionsToPackage{hyphens}{url}
%
\documentclass[
]{article}
\usepackage{amsmath,amssymb}
\usepackage{iftex}
\ifPDFTeX
  \usepackage[T1]{fontenc}
  \usepackage[utf8]{inputenc}
  \usepackage{textcomp} % provide euro and other symbols
\else % if luatex or xetex
  \usepackage{unicode-math} % this also loads fontspec
  \defaultfontfeatures{Scale=MatchLowercase}
  \defaultfontfeatures[\rmfamily]{Ligatures=TeX,Scale=1}
\fi
\usepackage{lmodern}
\ifPDFTeX\else
  % xetex/luatex font selection
\fi
% Use upquote if available, for straight quotes in verbatim environments
\IfFileExists{upquote.sty}{\usepackage{upquote}}{}
\IfFileExists{microtype.sty}{% use microtype if available
  \usepackage[]{microtype}
  \UseMicrotypeSet[protrusion]{basicmath} % disable protrusion for tt fonts
}{}
\makeatletter
\@ifundefined{KOMAClassName}{% if non-KOMA class
  \IfFileExists{parskip.sty}{%
    \usepackage{parskip}
  }{% else
    \setlength{\parindent}{0pt}
    \setlength{\parskip}{6pt plus 2pt minus 1pt}}
}{% if KOMA class
  \KOMAoptions{parskip=half}}
\makeatother
\usepackage{xcolor}
\usepackage[margin=1in]{geometry}
\usepackage{color}
\usepackage{fancyvrb}
\newcommand{\VerbBar}{|}
\newcommand{\VERB}{\Verb[commandchars=\\\{\}]}
\DefineVerbatimEnvironment{Highlighting}{Verbatim}{commandchars=\\\{\}}
% Add ',fontsize=\small' for more characters per line
\usepackage{framed}
\definecolor{shadecolor}{RGB}{248,248,248}
\newenvironment{Shaded}{\begin{snugshade}}{\end{snugshade}}
\newcommand{\AlertTok}[1]{\textcolor[rgb]{0.94,0.16,0.16}{#1}}
\newcommand{\AnnotationTok}[1]{\textcolor[rgb]{0.56,0.35,0.01}{\textbf{\textit{#1}}}}
\newcommand{\AttributeTok}[1]{\textcolor[rgb]{0.13,0.29,0.53}{#1}}
\newcommand{\BaseNTok}[1]{\textcolor[rgb]{0.00,0.00,0.81}{#1}}
\newcommand{\BuiltInTok}[1]{#1}
\newcommand{\CharTok}[1]{\textcolor[rgb]{0.31,0.60,0.02}{#1}}
\newcommand{\CommentTok}[1]{\textcolor[rgb]{0.56,0.35,0.01}{\textit{#1}}}
\newcommand{\CommentVarTok}[1]{\textcolor[rgb]{0.56,0.35,0.01}{\textbf{\textit{#1}}}}
\newcommand{\ConstantTok}[1]{\textcolor[rgb]{0.56,0.35,0.01}{#1}}
\newcommand{\ControlFlowTok}[1]{\textcolor[rgb]{0.13,0.29,0.53}{\textbf{#1}}}
\newcommand{\DataTypeTok}[1]{\textcolor[rgb]{0.13,0.29,0.53}{#1}}
\newcommand{\DecValTok}[1]{\textcolor[rgb]{0.00,0.00,0.81}{#1}}
\newcommand{\DocumentationTok}[1]{\textcolor[rgb]{0.56,0.35,0.01}{\textbf{\textit{#1}}}}
\newcommand{\ErrorTok}[1]{\textcolor[rgb]{0.64,0.00,0.00}{\textbf{#1}}}
\newcommand{\ExtensionTok}[1]{#1}
\newcommand{\FloatTok}[1]{\textcolor[rgb]{0.00,0.00,0.81}{#1}}
\newcommand{\FunctionTok}[1]{\textcolor[rgb]{0.13,0.29,0.53}{\textbf{#1}}}
\newcommand{\ImportTok}[1]{#1}
\newcommand{\InformationTok}[1]{\textcolor[rgb]{0.56,0.35,0.01}{\textbf{\textit{#1}}}}
\newcommand{\KeywordTok}[1]{\textcolor[rgb]{0.13,0.29,0.53}{\textbf{#1}}}
\newcommand{\NormalTok}[1]{#1}
\newcommand{\OperatorTok}[1]{\textcolor[rgb]{0.81,0.36,0.00}{\textbf{#1}}}
\newcommand{\OtherTok}[1]{\textcolor[rgb]{0.56,0.35,0.01}{#1}}
\newcommand{\PreprocessorTok}[1]{\textcolor[rgb]{0.56,0.35,0.01}{\textit{#1}}}
\newcommand{\RegionMarkerTok}[1]{#1}
\newcommand{\SpecialCharTok}[1]{\textcolor[rgb]{0.81,0.36,0.00}{\textbf{#1}}}
\newcommand{\SpecialStringTok}[1]{\textcolor[rgb]{0.31,0.60,0.02}{#1}}
\newcommand{\StringTok}[1]{\textcolor[rgb]{0.31,0.60,0.02}{#1}}
\newcommand{\VariableTok}[1]{\textcolor[rgb]{0.00,0.00,0.00}{#1}}
\newcommand{\VerbatimStringTok}[1]{\textcolor[rgb]{0.31,0.60,0.02}{#1}}
\newcommand{\WarningTok}[1]{\textcolor[rgb]{0.56,0.35,0.01}{\textbf{\textit{#1}}}}
\usepackage{graphicx}
\makeatletter
\def\maxwidth{\ifdim\Gin@nat@width>\linewidth\linewidth\else\Gin@nat@width\fi}
\def\maxheight{\ifdim\Gin@nat@height>\textheight\textheight\else\Gin@nat@height\fi}
\makeatother
% Scale images if necessary, so that they will not overflow the page
% margins by default, and it is still possible to overwrite the defaults
% using explicit options in \includegraphics[width, height, ...]{}
\setkeys{Gin}{width=\maxwidth,height=\maxheight,keepaspectratio}
% Set default figure placement to htbp
\makeatletter
\def\fps@figure{htbp}
\makeatother
\setlength{\emergencystretch}{3em} % prevent overfull lines
\providecommand{\tightlist}{%
  \setlength{\itemsep}{0pt}\setlength{\parskip}{0pt}}
\setcounter{secnumdepth}{-\maxdimen} % remove section numbering
\usepackage{booktabs}
\usepackage{longtable}
\usepackage{array}
\usepackage{multirow}
\usepackage{wrapfig}
\usepackage{float}
\usepackage{colortbl}
\usepackage{pdflscape}
\usepackage{tabu}
\usepackage{threeparttable}
\usepackage{threeparttablex}
\usepackage[normalem]{ulem}
\usepackage{makecell}
\usepackage{xcolor}
\ifLuaTeX
  \usepackage{selnolig}  % disable illegal ligatures
\fi
\IfFileExists{bookmark.sty}{\usepackage{bookmark}}{\usepackage{hyperref}}
\IfFileExists{xurl.sty}{\usepackage{xurl}}{} % add URL line breaks if available
\urlstyle{same}
\hypersetup{
  pdftitle={E-commerceDB},
  pdfauthor={group-8},
  hidelinks,
  pdfcreator={LaTeX via pandoc}}

\title{E-commerceDB}
\author{group-8}
\date{2024-02-27}

\begin{document}
\maketitle

\hypertarget{load-necessary-libraries}{%
\section{Load necessary libraries}\label{load-necessary-libraries}}

\begin{Shaded}
\begin{Highlighting}[]
\FunctionTok{library}\NormalTok{(knitr)}
\FunctionTok{library}\NormalTok{(kableExtra)}
\end{Highlighting}
\end{Shaded}

\hypertarget{part-1-database-design-and-implementation}{%
\section{Part 1: Database Design and
Implementation}\label{part-1-database-design-and-implementation}}

\hypertarget{entity-relationship-diagram}{%
\subsection{1.1 Entity Relationship
Diagram}\label{entity-relationship-diagram}}

\begin{itemize}
\tightlist
\item
  Here Insert ERD *
\end{itemize}

The E-R diagram above simulates a real-world e-commerce data ecosystem,
capturing the detailed relationships between entities and attributes
essential for facilitating online transactions. In addition, it provides
a comprehensive view of the e-commerce system, which serves as a
platform for users to browse products, make purchases, and securely
complete their payments.

\hypertarget{assumptions}{%
\subsubsection{1.1.1.Assumptions}\label{assumptions}}

\begin{itemize}
\item
  The company only distributes products within the United Kingdom (UK).
\item
  The Currency used is Pound Sterling (GBP).
\item
  Attributes formats will be aligned with UK standard formats such as
  date , addresses , names \ldots etc
\end{itemize}

\hypertarget{entities-and-attributes}{%
\subsubsection{1.1.2. Entities and
Attributes}\label{entities-and-attributes}}

This section describes and illustrates the entities and their respective
attributes of the ER diagram above. \#\#\#\# 1.1.2.1. Customer Users who
previously have at least once purchased products and placed an order.
\#\#\#\# 1.1.2.2. Supplier Vendors who provide products. Represent the
source of the product items.

\hypertarget{product}{%
\paragraph{1.1.2.3. Product}\label{product}}

\begin{verbatim}
Indicates all products in the stock and available for sale.
\end{verbatim}

\hypertarget{design-considerations}{%
\section{Design Considerations}\label{design-considerations}}

\begin{itemize}
\item
  Absence of an Order Entity:
\item
  The model intentionally skips direct order management. Instead, it
  focuses on product management and customer interactions through
  reviews and payment methods.Additionally, This consideration will
  guarantee that products purchased by customers are not tracked or
  stored by the system to align with privacy policies.
\item
  Order Entity not considered in this ER design in order to follow best
  practices by not having to include orderId as part of product table
  which might affect the overall performance of DB retrieval.
\item
  Customer Engagement: By including Reviews, the model emphasizes
  customer engagement and feedback without directly managing
  transactions.
\item
  Payment Information: Including Payment\_Method without an Order entity
  suggests a pre-registration of payment preferences or a simplified
  wallet storage that could be expanded in the future.
\end{itemize}

\hypertarget{logical-schema}{%
\subsection{Logical Schema}\label{logical-schema}}

Customers (\underline{Customer_id}, Customer\_Email, Cust\_F\_Name,
Cust\_L\_Name, Phone\_Country\_Code, Phone\_Num, Cust\_Street\_Name,
Cust\_Building\_Name, Cust\_Zip\_Code)

Products (Product\_id, Discount\_id, Category\_id, Product\_Name,
Product\_Price, Product\_Availability)

Suppliers (Supplier\_id, Supplier\_Email, Supplier\_Name,
Supplier\_Status, Sup\_Building\_Name, Sup\_Street\_Name, Sup\_
Zip\_Code)

Order\_Details (Order\_id, Customer\_id, Order\_Date, Order\_Total,
Order\_Status, S\_Building\_Name, S\_Street\_Name, S\_Zip\_Code,
Street\_Name, B\_Building\_Name, B\_Street\_Name, B\_Zip\_Code,
Payment\_Type, Payment\_Status)

Discounts (Discount\_Code, Discount\_Status, Discount\_Amount) Reviews
(Review\_id, Product\_id, Review\_Rating, Review\_Timestamp,
Review\_Text, Review\_Likes)

Categories (Category\_id, Category\_Name)

Order\_Items ( Order\_id, Customer\_id, Product\_id, Quantity,
Unit\_Price)

Supplied\_Items (Supplier\_id, Product\_id, Supply\_Contracts,
Delivery\_Terms, Pricing\_Agreements)

Many to Many relation Table of Supplier and Product SupplierProduct
(Supplier\_id, Product\_id)

Many to Many relation Table of Order\_details and Product
ProductOrder\_details (Order\_id, Product\_id)

\end{document}
